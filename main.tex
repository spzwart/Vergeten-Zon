%%%%%%%%%%%%%%%%%%%%%%%%%%%%%%%%%%%%%%%%%
%  My documentation report
%  Objetive: Explain what I did and how, so someone can continue with the investigation
%
% Important note:
% Chapter heading images should have a 2:1 width:height ratio,
% e.g. 920px width and 460px height.
%
%%%%%%%%%%%%%%%%%%%%%%%%%%%%%%%%%%%%%%%%%

%----------------------------------------------------------------------------------------
%	PACKAGES AND OTHER DOCUMENT CONFIGURATIONS
%----------------------------------------------------------------------------------------

\documentclass[11pt,fleqn]{book} % Default font size and left-justified equations

\usepackage[top=3cm,bottom=3cm,left=3.2cm,right=3.2cm,headsep=10pt,letterpaper]{geometry} % Page margins

\usepackage{url}
\usepackage{xcolor} % Required for specifying colors by name
\definecolor{ocre}{RGB}{52,177,201} % Define the orange color used for highlighting throughout the book

% Font Settings
\usepackage{avant} % Use the Avantgarde font for headings
%\usepackage{times} % Use the Times font for headings
\usepackage{mathptmx} % Use the Adobe Times Roman as the default text font together with math symbols from the Sym­bol, Chancery and Com­puter Modern fonts

\usepackage{microtype} % Slightly tweak font spacing for aesthetics
\usepackage[utf8]{inputenc} % Required for including letters with accents
\usepackage[T1]{fontenc} % Use 8-bit encoding that has 256 glyphs

% Bibliography
\usepackage[style=alphabetic,sorting=nyt,sortcites=true,autopunct=true,babel=hyphen,hyperref=true,abbreviate=false,backref=true,backend=biber]{biblatex}
\addbibresource{bibliography.bib} % BibTeX bibliography file
\defbibheading{bibempty}{}

\input{structure} % Insert the commands.tex file which contains the majority of the structure behind the template

\begin{document}

%----------------------------------------------------------------------------------------
%	TITLE PAGE
%----------------------------------------------------------------------------------------

\begingroup
\thispagestyle{empty}
\AddToShipoutPicture*{\put(0,0){\includegraphics[scale=1.25]{Sun.png}}} % Image background
\centering
\vspace*{5cm}
\par\normalfont\fontsize{35}{35}\sffamily\selectfont
\textbf{Ons eenzame zonnestelsel}\\
{\LARGE Over de vragen die bleven hangen}\par % Book title
\vspace*{1cm}
{\Huge Martijn van Calmthout \\
       Simon Portegies Zwart}\par % Author name
\endgroup

%----------------------------------------------------------------------------------------
%	COPYRIGHT PAGE
%----------------------------------------------------------------------------------------

\newpage
~\vfill
\thispagestyle{empty}

%\noindent Copyright \copyright\ 2014 Andrea Hidalgo\\ % Copyright notice

\noindent \textsc{Summer Research Internship, University of Western Ontario}\\

\noindent \textsc{github.com/LaurethTeX/Clustering}\\ % URL

\noindent Dit boek is tot stand gekomen door een samenwerking tussen Martijn van Calmthout en Simon Portegies Zwart, geinitieerd door Liesbeth.\\ % License information

\noindent \textit{Begonnen met schrijven in oktober 2018} % Printing/edition date

%----------------------------------------------------------------------------------------
%	TABLE OF CONTENTS
%----------------------------------------------------------------------------------------

\chapterimage{solarsystem.png} % Table of contents heading image

\pagestyle{empty} % No headers

\tableofcontents % Print the table of contents itself

%\cleardoublepage % Forces the first chapter to start on an odd page so it's on the right

\pagestyle{fancy} % Print headers again

%----------------------------------------------------------------------------------------
%	CHAPTER 1
%----------------------------------------------------------------------------------------

\chapterimage{ejected_planet_940x500.png} % Chapter heading image

\chapter{Introductie}

\chapter{Voor het zonnestelsel}

Wat gebeurde er voordat er een zonnestelsel was? Ik denk dat we hier
kort in kunnen gaan op wat er voor het zonnestelsel was. zoeits als:

Het heelal is niet begonnen met het zonnestelsel, maar was al 9
miljard jaar oud toen de zon werd geboren. In die periode is er een
boel gebeurd. volgens de standaard tehorie over het ontstaan van het
heelal is het geboren met de chemische elementen waterstof, helium en
lithium. Maar op de aarde vinden we bijvoorbeel ook Koolstof,
Aluminium, ijzer en goud. (ik noem hier Al en Fe, omdat we daar later
op terug kunnen komen, bij de curieuze verrijking van het zonnestelsel
in die elementen).

Vervolgens zouden we de hangbare theorie kunnen bespreken, om daarna de problemen aan te kaarten.
Onderwerpen kunnen zijn:
\section{Het ontstaan van dingen: donkere energie en donkere materie}
Komen we weg zonder donkere dingen te noemen? Dat zou eea wel vereenvoudigen.

\section{het ontstaan van de chemische elementen}
We zouden het kort kunnen hebben over het ontstaan van Aluminium,
ijzer, zilver en goud, etc.  Het aardige hierbij zal zijn, dat we het
later over diamanten kunnen hebben, die dus niet in sterren worden
gevormd, maar in planeten. Maar om diamanten te maken heb je wel
koolstof nodig, dat weer uit sterren komt. Ik denk dat we daar wel
iets aanstekelijks over kunnen zeggen.

De eerste sterren ontstonden 180 miljoen jaar na het ontstaan van het
heelal. Deze sterren waren mogelijk nogal
zwaar en werden geboren in sterren hopen. Maar heel veel weten we er
niet van. Schattingen voor hun massa lopen uiteen van zo zwaar als de
zon tot 1000 keer zo zwaar. Als ze zwaar zijn, dan branden ze snel op,
maar heel lichte sterren kunnen nog altijd bestaan. Het zou mogelijk
zijn, dat er nog enekele van die eerste sterren in het melkwegstelsel
aanwezig zijn. Een voorbeeld is: 

SMSS J031300.36-670839.3 also known as Keller's star is, met een
leeftijd van 13.6 miljard jaar, de oudst bekende ster.  heeft 400 keer
minder koolstof dan de zon, en 10 miljoen keer minder ijzer.  Deze
ster is gevormd kort nadat een eerste generatie sterren het
interestellaire materiaal heeft verrijkt. Aangezien er wel koolstof en ijzer in de ster zit, betekend dit dat deze ster niet echt een 'eersteling' is, maar al weer een 2e of 3e generatie sterren. Het oorspronkelijke koolstof en ijzer is namelijk in een eerdere generatie sterren gevormd.  Er zijn nogal wat vragen te stellen over dergelijke eerste ster populaties, maar ik denk niet dat we daar heel diep op in willen gaan.

\begin{figure}[h]
    \centering
    \includegraphics[width=0.37\textwidth]{Pictures/OldestStar-SM0313-SMSSJ031300366708393-20140210.png}
    \caption{Opname van het veld om de oudst bekende ster SMSS
      J031300.36-670839.3.  Gefotografeerd op 10 februarie 2014 met
      het Sky Mapper instrument in Siding Spring Observatorium in
      Australie.  (beeld afkomstig van NASA, van Wikipedia:
      \url{https://en.wikipedia.org/wiki/SMSS_J031300.36-670839.3}.
      copy right \url{http://www.stsci.edu/institute/Copyright}.) }
    \label{fig:pca}
\end{figure}

In het kort: chemische elementen worden gevormd in drie processen: de
wind uitstoot van sterren, supernovae en botsende neutronen sterren.
Wind levert vooral tot Aluminium, supernovae levert tot zgn S-processed elementen (zie Fig. \ref{fig:periodictable}, daarna heb je nog exotiser processen nodig.

Lastige is natuurlijk dat we niet de hele ster evolutie willen
behandelen.  Interesante vragen zijn: Kun je aan de verhouding van
zware elementen alfezen wat er precies is gebeurd, en in welke
volgorde?

\begin{figure}[h]
    \centering
    \includegraphics[width=0.37\textwidth]{Pictures/periodictable.png}
    \caption{Periodieke tabel van de elementen, met aanduiding waar ze vandaan komen(beeld afkomstig van Wikipedia:
      \url{https://en.wikipedia.org/wiki/S-process}.
      }
    \label{fig:periodictable}
\end{figure}

\section{Vorming van de Melkweg}

De zon is onderdeel van de Melkweg, en ik denk dat we hier iets over
moeten zeggen al is het maar kort.  De eerste sterren vormden in
groepen, die vervolgens groeiden door het samenwmelten van
groepen. Hierdoor ontstond een wolk van sterren, die door
draaiimpulsmoment een afgeplatte vorm met een centrale verdikking
kreeg. Zo is de melkweg geboren.

Vragen die overblijven: spriaal structuur, de kern, is die bal- of
pinda-vormig. De centrale balk, hoe is die ontstaan. Wat is de rol van
donkere materie in de vorming. Kunnen we botsingen met andere
sterrenstelsel nog herkennen. Zeker ook iets over Gaia?

Ik denk dat we een dergelijke introductie van de beschrijving van de
omgeving nodig hebben om het podium te schetsen. Later zullen we op
deze zaken terug komen.


\chapter{Het ontstaan van het zonnestelsel}
\chapter{Consequenties van geboorte in een sterrenhoop}
\chapter{Het eerste licht: de vlam gaat aan}
\chapter{De eerste planeten}
\chapter{De Maan}
\chapter{De late evolutie en de beweging van de planeten}
\chapter{Problemen van vandaag}
\chapter{Wat er nog gaat gebeuren}

\section{Openstaande vragen}\index{Vragen}

   
\begin{itemize} 
\item voor de zon er was, was er gas
\item De vorming van de Zon en de gasdisk er omheen
\item De geboorteomgeving van de Zon, in een sterrenhoop.
\item De consequenties van geboren worden in een sterrenhoop:
  \begin{itemize} 
  \item    -verrijking door sterrenwing
  \item    -supernova in de buurd (disk afkapping, en omduwen, roosteren van de schijf)
  \end{itemize} 
\item De verdamping van de geboorte cluster, en waar die sterren nu rondhangen (brusjes v/d zon)
\item Het ontstaan van de Oortwolk en de oorspong van de kometen
\item De periode van het late-zware bombardement en de oorsprong van de maankraters (Nice model)
\item De verdere evolutie van de zon, als een eenzame ster.
\item De vorming van de planeten en manen
\item Het overblijvende materiaal (de planetoiden)
\item Mixing van het overblijvende materiaal (Overstag [grand-tack] model vs planeet embrio model)
\item Het einde van het leven op aarde (door een meteoriet of komeetinslag).
\item Het sterven van de zon (over ca. 5 miljar jaar)
\end{itemize}

\begin{itemize}
\item Open staande en uitgesproken vragen:
  \begin{itemize}
  \item -Waar komt het waten op aarde (en Mars) vandaan?
  \item -Waarom heeft Jupiter zo veel manen, en wat is de rol van de Trojanen
  \item -Waarom ligt Uranus op zijn kant
  \item -Hoe komt Pluto aan zo veel manen
  \item -Wat is de oorsprong van de vreemde baan van Sedna
  \item -Hoe vaak botsen er asteroide of planetoiden op de aarde
  \item -Waarom is de zon veel rijker aan metalen dan omliggende sterren.
  \item -Hoe komt het dat het zonnestelsel zo'n 'af' planetenstelsel heeft.
  \end{itemize}
\end{itemize}

\subsection{Vragen van MvC aan SPZ}

\begin{itemize}
  \item Hoe we veel minder van zon en planeten begrijpen dan we meestal denken, en dat de astronomie niet echt kan schelen

  \item   Het lijkt zo vertrouw, het zonnestelsel. Een handvol planeten die ordelijk om een centrale ster draaien, sommige van rosten, andere van gas. Wat manen. Een enkele komeet.

  \item   We snappen de bewegingen, de geschiedenis van zon en planeten, we hebben een vermoeden van de toekomst. Er zijn foto’s, verkenners en zelfs landers die op buurplaneten rondrijden.

  \item   Bekend terrein, denken we. In werkelijkheid wemelt het zonnestelsel van de raadsels en onbegrepen verschijnselen. Waar komt de zon vandaan? Waarom ligt Neptunus op zijn kant? Waar komen vreemde stenen op aarde vandaan? Wat zijn de ruimtekeien die nu en dan in de verte passeren?

  \item   In Het Raadsel Zonnestelsel verkennen schrijver Martijn van Calmthout en astrofysicus Simon Portegies Zwart de onbekende kanten van de ster en planeten om ons heen. In een tijd dat de astronomie gretig jaagt op de oerknal en planeten bij verre sterren, lijken we onze kosmische voortuin wat te verwaarlozen.
\end{itemize}
  

\begin{quote}
haha..
\end{quote}

\begin{figure}[h]
    \centering
    \includegraphics[width=0.37\textwidth]{TheSun.png}
    \caption{leuk plaatje}
    \label{fig:pca}
\end{figure}



%----------------------------------------------------------------------------------------
%	CHAPTER 3
%----------------------------------------------------------------------------------------

\chapterimage{solarsystem.png}
\chapter{Hoofdstuk 2...}

bladiebla...

\section{sectie ... ?}


\begin{table}[h!]
  \centering
    \begin{tabular}{ c c c c c c }
    \hline\hline
    
    Name & Input nodes & Normalized data & Learning rate & Epsilon & Pruning Frequency\\
    \hline
    
    Train2 & 1 & 1 & 0.3 & 0.001 & 5\\
    Train3 & 1 & 1 & 0.7 & 10 & 100\\
    Train4 & 1 & 1 & 0.95 & 1 & 10\\
    Train5 & 1 & 1 & 0.99 & 0.1 & 10\\
    Train6 & 1 & 1 & 0.01 & 0.01 & 1\\
    Train7 & 1 & 1 & 0.5 & 0.7 & 5\\
    Train8 & 1 & 1 & 0.5 & 0.5 & 7\\
    Train11 & 1 & 1 & 0.25 & 0.00001 & 10\\
    
    \hline
  \end{tabular}
  \caption{tabel voorbeeld..}
  \label{tab:ds9failed}
\end{table}

\cite{book_key}

%% \bibliographystyle{} % Style BST file
%% \bibliography{bibliography}

\end{document}
