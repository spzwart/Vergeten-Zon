%%%%%%%%%%%%%%%%%%%%%%%%%%%%%%%%%%%%%%%%%
%  My documentation report
%  Objetive: Explain what I did and how, so someone can continue with the investigation
%
% Important note:
% Chapter heading images should have a 2:1 width:height ratio,
% e.g. 920px width and 460px height.
%
%%%%%%%%%%%%%%%%%%%%%%%%%%%%%%%%%%%%%%%%%

%----------------------------------------------------------------------------------------
%	PACKAGES AND OTHER DOCUMENT CONFIGURATIONS
%----------------------------------------------------------------------------------------

\documentclass[11pt,fleqn]{book} % Default font size and left-justified equations

\usepackage[top=3cm,bottom=3cm,left=3.2cm,right=3.2cm,headsep=10pt,letterpaper]{geometry} % Page margins

\usepackage{xcolor} % Required for specifying colors by name
\definecolor{ocre}{RGB}{52,177,201} % Define the orange color used for highlighting throughout the book

% Font Settings
\usepackage{avant} % Use the Avantgarde font for headings
%\usepackage{times} % Use the Times font for headings
\usepackage{mathptmx} % Use the Adobe Times Roman as the default text font together with math symbols from the Sym­bol, Chancery and Com­puter Modern fonts

\usepackage{microtype} % Slightly tweak font spacing for aesthetics
\usepackage[utf8]{inputenc} % Required for including letters with accents
\usepackage[T1]{fontenc} % Use 8-bit encoding that has 256 glyphs

% Bibliography
\usepackage[style=alphabetic,sorting=nyt,sortcites=true,autopunct=true,babel=hyphen,hyperref=true,abbreviate=false,backref=true,backend=biber]{biblatex}
\addbibresource{bibliography.bib} % BibTeX bibliography file
\defbibheading{bibempty}{}

\input{structure} % Insert the commands.tex file which contains the majority of the structure behind the template

\begin{document}

%----------------------------------------------------------------------------------------
%	TITLE PAGE
%----------------------------------------------------------------------------------------

\begingroup
\thispagestyle{empty}
\AddToShipoutPicture*{\put(0,0){\includegraphics[scale=1.25]{Sun.png}}} % Image background
\centering
\vspace*{5cm}
\par\normalfont\fontsize{35}{35}\sffamily\selectfont
\textbf{De vergeten Zon}\\
{\LARGE Over de vragen die bleven hangen}\par % Book title
\vspace*{1cm}
{\Huge Martijn van Calmthout \\
       Simon Portegies Zwart}\par % Author name
\endgroup

%----------------------------------------------------------------------------------------
%	COPYRIGHT PAGE
%----------------------------------------------------------------------------------------

\newpage
~\vfill
\thispagestyle{empty}

%\noindent Copyright \copyright\ 2014 Andrea Hidalgo\\ % Copyright notice

\noindent \textsc{Summer Research Internship, University of Western Ontario}\\

\noindent \textsc{github.com/LaurethTeX/Clustering}\\ % URL

\noindent Dit boek is tot stand gekomen door een samenwerking tussen Martijn van Calmthout en Simon Portegies Zwart, geinitieerd door Liesbeth.\\ % License information

\noindent \textit{Begonnen met schrijven in oktober 2018} % Printing/edition date

%----------------------------------------------------------------------------------------
%	TABLE OF CONTENTS
%----------------------------------------------------------------------------------------

\chapterimage{solarsystem.png} % Table of contents heading image

\pagestyle{empty} % No headers

\tableofcontents % Print the table of contents itself

%\cleardoublepage % Forces the first chapter to start on an odd page so it's on the right

\pagestyle{fancy} % Print headers again

%----------------------------------------------------------------------------------------
%	CHAPTER 1
%----------------------------------------------------------------------------------------

\chapterimage{ejected_planet_940x500.png} % Chapter heading image

\chapter{Introduction}

\section{Openstaande vragen}\index{Vragen}

   
\begin{itemize} 
\item voor de zon er was, was er gas
\item De vorming van de Zon en de gasdisk er omheen
\item De geboorteomgeving van de Zon, in een sterrenhoop.
\item De consequenties van geboren worden in een sterrenhoop:
  \begin{itemize} 
  \item    -verrijking door sterrenwing
  \item    -supernova in de buurd (disk afkapping, en omduwen, roosteren van de schijf)
  \end{itemize} 
\item De verdamping van de geboorte cluster, en waar die sterren nu rondhangen (brusjes v/d zon)
\item Het ontstaan van de Oortwolk en de oorspong van de kometen
\item De periode van het late-zware bombardement en de oorsprong van de maankraters (Nice model)
\item De verdere evolutie van de zon, als een eenzame ster.
\item De vorming van de planeten en manen
\item Het overblijvende materiaal (de planetoiden)
\item Mixing van het overblijvende materiaal (Overstag [grand-tack] model vs planeet embrio model)
\item Het einde van het leven op aarde (door een meteoriet of komeetinslag).
\item Het sterven van de zon (over ca. 5 miljar jaar)
\end{itemize}

\begin{itemize}
\item Open staande en uitgesproken vragen:
  \begin{itemize}
  \item -Waar komt het waten op aarde (en Mars) vandaan?
  \item -Waarom heeft Jupiter zo veel manen, en wat is de rol van de Trojanen
  \item -Waarom ligt Uranus op zijn kant
  \item -Hoe komt Pluto aan zo veel manen
  \item -Wat is de oorsprong van de vreemde baan van Sedna
  \item -Hoe vaak botsen er asteroide of planetoiden op de aarde
  \item -Waarom is de zon veel rijker aan metalen dan omliggende sterren.
  \item -Hoe komt het dat het zonnestelsel zo'n 'af' planetenstelsel heeft.
  \end{itemize}
\end{itemize}

\begin{quote}
haha..
\end{quote}

\begin{figure}[h]
    \centering
    \includegraphics[width=0.37\textwidth]{TheSun.png}
    \caption{leuk plaatje}
    \label{fig:pca}
\end{figure}



%----------------------------------------------------------------------------------------
%	CHAPTER 3
%----------------------------------------------------------------------------------------

\chapterimage{solarsystem.png}
\chapter{Hoofdstuk 2...}

bladiebla...

\section{sectie ... ?}


\begin{table}[h!]
  \centering
    \begin{tabular}{ c c c c c c }
    \hline\hline
    
    Name & Input nodes & Normalized data & Learning rate & Epsilon & Pruning Frequency\\
    \hline
    
    Train2 & 1 & 1 & 0.3 & 0.001 & 5\\
    Train3 & 1 & 1 & 0.7 & 10 & 100\\
    Train4 & 1 & 1 & 0.95 & 1 & 10\\
    Train5 & 1 & 1 & 0.99 & 0.1 & 10\\
    Train6 & 1 & 1 & 0.01 & 0.01 & 1\\
    Train7 & 1 & 1 & 0.5 & 0.7 & 5\\
    Train8 & 1 & 1 & 0.5 & 0.5 & 7\\
    Train11 & 1 & 1 & 0.25 & 0.00001 & 10\\
    
    \hline
  \end{tabular}
  \caption{tabel voorbeeld..}
  \label{tab:ds9failed}
\end{table}

\cite{book_key}

%% \bibliographystyle{} % Style BST file
\bibliography{bibliography}  
\end{document}
